\documentclass[conference]{IEEEtran}

\ifCLASSINFOpdf
  \usepackage[pdftex]{graphicx}
\else
  \usepackage[dvips]{graphicx}
\fi

\hyphenation{op-tical net-works semi-conduc-tor}

% Document.
\begin{document}

\title{Comparative analysis of performance using server-client protocols}

\author{\IEEEauthorblockN{Mihail Costea}
\IEEEauthorblockA{University Politehnica of Bucharest\\
The Faculty of Automatic Control and Computers\\
Bucharest, Romania\\
Email: mihail.costea90@gmail.com}
\and
\IEEEauthorblockN{Liviu Chircu}
\IEEEauthorblockA{University Politehnica of Bucharest\\
The Faculty of Automatic Control and Computers\\
Bucharest, Romania\\
Email: liviu.chircu@gmail.com}}

\maketitle

\begin{abstract}
These days most popular solutions for bi-directional communication in
web applications are based on AJAX.
Even though they are well documented solutions that are backed up by years of
utilization, they have limitations imposed by the HTTP protocol. HTTP is a
stateless protocol that requires each connection to be threated as a new
connection, requiring unnecessary overhead to communicaty in both directions.
Because of these limitations a new solution was developed, WebSockets,
which are able to natively mantain a bi-directional channel, reducing the
overhead needed for communication.
\\
\indent
This paper proposes to exemplify the advantages and disadvantages between
traditional HTTP implementations for bi-directional cummunication based on AJAX
and WebSockets. Also it proposes an arhitecture for a testing platform for
different WebSockets implementations.
\end{abstract}

\IEEEpeerreviewmaketitle

% Content.
\section{Introduction}
Wireless communication is the transfer of information
between two or more points that are not connected
by an electrical conductor.
 
\hfill Octomber 22, 2013

\section{Related work}
The most common wireless technologies use electromagnetic
wireless telecommunications, such as radio. \cite{3}

\subsection{Work 1}
With radio waves distances can be short, such as a few meters
for television or as far as thousands or even millions of
kilometers for deep-space radio communications.

\subsubsection{Subsection 1}
It encompasses various types of fixed, mobile, and
portable applications, including two-way radios, cellular
telephones, personal digital assistants (PDAs), and
wireless networking. Other examples of applications 
of radio wireless technology include GPS units, garage 
door openers, wireless computer mice, keyboards and headsets,
 headphones, radio receivers, satellite television,
broadcast television and cordless telephones.

\subsubsection{Subsection 2}
It encompasses various types of fixed, mobile, and
portable applications, including two-way radios, cellular
telephones, personal digital assistants (PDAs), \cite{2} and
wireless networking. Other examples of applications 
of radio wireless technology include GPS units, garage 
door openers, wireless computer mice, keyboards and headsets,
 headphones, radio receivers, satellite television,
broadcast television and cordless telephones.

\subsection{Work 2}
Less common methods of achieving wireless communications
include the use of light, sound, magnetic, or electric fields.

\section{Architecture/Design}
The most common wireless technologies use electromagnetic
wireless telecommunications, such as radio.

\subsection{Subsection}
Less common methods of achieving wireless communications
include the use of light, sound, magnetic, or electric fields.

\section{Implementation}
The most common wireless technologies use electromagnetic
wireless telecommunications, such as radio.

\section{Conclusion}
Wireless energy transfer is a process whereby electrical
energy is transmitted from a power source to an electrical
load that does not have a built-in power source, without
the use of interconnecting wires.

\section*{Acknowledgment}

The authors would like to thank...


\bibliography{bare_conf}{}
\bibliographystyle{unsrt}

\end{document}


